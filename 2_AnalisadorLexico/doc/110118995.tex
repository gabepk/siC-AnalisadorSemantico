\documentclass[12pt]{article}
\usepackage[T1]{fontenc}
\usepackage{float}
\usepackage{sbc-template}
\usepackage{graphicx,url}
\usepackage[brazil]{babel}   
\usepackage[utf8]{inputenc}  
\usepackage{indentfirst}
\usepackage{caption3}
\usepackage{color}
\usepackage{alltt}
\usepackage{url}
\usepackage{listings}

\lstset{ %
language=C,                % choose the language of the code
basicstyle=\footnotesize,       % the size of the fonts that are used for the code
numbers=left,                   % where to put the line-numbers
numberstyle=\footnotesize,      % the size of the fonts that are used for the line-numbers
stepnumber=1,                   % the step between two line-numbers. If it is 1 each line will be numbered
numbersep=7pt,                  % how far the line-numbers are from the code
backgroundcolor=\color{white},  % choose the background color. You must add \usepackage{color}
showspaces=false,               % show spaces adding particular underscores
showstringspaces=false,         % underline spaces within strings
showtabs=false,                 % show tabs within strings adding particular underscores
frame=single,           % adds a frame around the code
tabsize=2,          % sets default tabsize to 2 spaces
breaklines=true,        % sets automatic line breaking
breakatwhitespace=false,    % sets if automatic breaks should only happen at whitespace
escapeinside={\%*}{*)},          % if you want to add a comment within your code
extendedchars=true,
emph={%  
    set, pair, bool%
    },emphstyle={\bfseries},
morekeywords={set, pair}
literate={á}{{\'a}}1 {ã}{{\~a}}1 {é}{{\'e}}1
}
     
\sloppy

\title{siC: Uma linguagem baseada em C incluindo fila como tipo primitivo}

\author{Gabriella de Oliveira Esteves, 110118995}

\address{Departamento de Ciência da Computação - Universidade de Brasília}

\begin{document} 

\maketitle

%------------------------------------------------
\section{Objetivo}

Este trabalho visa projetar e construir uma nova linguagem chamada de siC - Structure in C, baseada na linguagem C. O siC acrescenta a estrutura de dados fila como tipo de dado primitivo e, para manipulá-la, adiciona certas operações próprias para tal.

%------------------------------------------------
\section{Introdução}

\indent Um compilador é um programa que recebe como entrada um código fonte e o traduz para um programa equilavente em outra linguagem \cite{book}. Ele pode ser dividido em sete fases, ilustrado na Figura \ref{fig:compilador}.

\begin{figure}[!ht]
  \centering
  \includegraphics[width=0.5\textwidth]{compilador.png}
  \caption{Fases de um compilador} \label{fig:compilador}
\end{figure}

\begin{itemize}
	\item[1] \textbf{Analisador Léxico:} Lê o código fonte e atribui significado à cada sequência de caracteres, agora chamados lexemas. Cada lexema é mapeado para um token, que por sua vez é um par de nome (símbolo abstrato) e atributo (ponteiro para tabela de símbolos);
	\item[2] \textbf{Analisador Sintático:} Constrói uma representação gramátical dos tokens em forma de árvore;
	\item[3] \textbf{Analisador Semântico:} Utiliza a árvore sintática juntamente com a tabela de símbolos para verificar se a consistência semântica é mantida de acordo com a definição da linguagem.
	\item[4] \textbf{Gerador de Código Intermediário:} Converte árvore sintática anotada em código intermediário, com linguagem parecida com assembly e que possui apenas três operadores por linha de código. Nesse sentido, quebra-se estruturas complexas em estruturas mais simples, nesta fase.
	\item[5, 6] \textbf{Otimizador de Código Independente/Dependente de Máquina:} Procura aprimorar o código intermediário com o objetivo de melhorar o código-alvo de alguma forma: o deixando mais rápido, mais curto, consumindo menos energia, etc.
	\item[7] \textbf{Gerador de Código:} Converte o código intermediário no código-alvo, 	buscando atribuir os registradores às variáveis da maneira ótima.
\end{itemize}

\indent O foco do projeto será nas fase 1, 2, 3 e 4, porém a princípio serão apresentadas apenas a descrição da linguagem siC, uma breve descrição de sua semântica e o analisador léxico. Como a fila é uma das estruturas de dados mais básicas, é possível dizer que siC se destina a inúmeras áreas de Ciência da Computação, como, por exemplo, sistemas operacionais, onde ela é usada para organizar prioridades dos processos.

\indent Dois grandes motivos sustentam a escolha do tema deste projeto. Primeiro, uma vez que a fila faz parte dos tipos primitivos de uma linguagem, haverá menos manipulação de ponteiros na mesma, portanto erros envolvendo-os são menos prováveis de ocorrer. Segundo, a linguagem siC é mais alto-nível que C devido à abstração desta estrutura de dados básicas, e, de maneira geral, pode ser mais \textit{user-friendly}. Nesse sentido, o usuário (da linguagem) leigo deverá entender como a estrutura funciona, bem como suas vantagens/desvantagens e usabilidade; porém a implementação de cada uma estará a cargo da própria siC. 

\section{Gramática}

\indent A seguir será apresentada a gramática da linguagem siC, baseada em C \cite{yacc}. Alguns comentários são feitos ao longo da gramática para facilitar o entendimento das variáveis e nomenclatura utilizada. As palavras reservadas da linguagen são representadas aqui como \textit{tokens}. As variáveis e constantes são representadas como \textit{identifiers}, que por sua vez é uma expressão regular, e a única diferença entre este e \textit{identifier\_struct} é que o segundo tem acesso ao início da fila caso este seja o tipo do \textit{identifier}. 

\indent Algumas alterações e correções foram feitas na gramática:

\begin{itemize}
	\item Será implementado apenas a fila como tipo primitivo. Se fosse mantida a primeira proposta de incluir também o tipo pilha, talvez não seria possível terminar o projeto no prazo previsto;
	\item O tipo booleano deixará de existir em siC, porém o tipo float será incluído;
	\item Agora, além da variável \textit{argument}, também existe a \textit{arguments}, que permite a definição de zero ou mais argumentos em uma função;
	\item Na primeira versão da gramática o \textit{statement} estava envolvido entre chaves no IF e no WHILE, enquanto nesta versão as chaves não são mais obrigatórias, porém a regra \textit{statement $\rightarrow$ \{ statement \}} foi adicionada para criação de blocos;
	\item Expressões matemáticas são agora da forma \textit{identifier $\rightarrow$ assignment\_expression} ao invés de \textit{identifier $\rightarrow$ factor} para maior legibilidade;
	\item As aspas que delimitavam os símbolos como chaves e parênteses foram retiradas para aumentar também a legibilidade;
	\item Foram adicionados as operações de comparação \textit{<} e \textit{>};
	\item O símbolo \$ foi adicionado na variável \textit{letra}, pois em C ele pode compor um identificador;
	\item A variavel \textit{caractere} foi adicionada para representar o valor de um char, que só poderá ser ou uma letra ou um dígito;
	\item A variável inicial será agora \textit{program}, que possibilitará a criação de uma ou mais funções no código fonte;
	\item Como existe agora chamada de função, foi criado mais um \textit{statement} para tal, \textit{identifier(identifiers);}, que exigiu a criação da nova variável \textit{identifiers} para a passagem de parâmetros O retorno das funções pode ser atribuído para uma variável, portanto também foi criada a regra \textit{identifier = identifier(identifiers);}.
\end{itemize}

Segue abaixo a gramática proposta cuja variável inicial é \textit{program}, com as alterações acima em vermelho e as características diferenciais da linguagem siC em negrito.
\begin{alltt}{\footnotesize

token: WHILE, IF, ELSE, RETURN 
token: QUEUE, FIRST, VOID, {\color{red}FLOAT}, INT, CHAR

{\color{red}program
   \(\to\) program function
   | function}

function
   \(\to\) {\color{red}argument} ( arguments ) \{ statement RETURN identifier ; \}
    
{\color{red}identifiers
   \(\to\) identifiers, identifier
   | identifier
   | \(\varepsilon\)}

identifier
   \(\to\) letra(letra | digito)*
   
{\color{red}caractere
   \(\to\) letra | digito}
    
letra
   \(\to\) a | b | \dots | z | A | B | \dots | Z | \$
   
digito
   \(\to\) 0 | 1 | \dots | 9
	
identifier\_struct
   \(\to\) identifier
    | \textbf{identifier . FIRST}	
	
type\_struct
   \(\to\) type\_simple
    | \textbf{type\_queue}
    
type\_simple
   \(\to\) VOID | {\color{red}FLOAT} | INT | CHAR
   
}\end{alltt}
\indent Existe um novo tipo de dado, \textit{QUEUE}, que será composto por tipos simples de dados apenas (ou seja, não será possível criar uma variável do tipo fila em que seus elementos também são filas). Caso a variável seja do tipo fila, ela poderá obter o primeiro elemento através do comando "identifier.FIRST".
\begin{alltt}{\footnotesize      

\textbf{type\_queue
   \(\to\) QUEUE < type\_simple >}

{\color{red}arguments
   \(\to\) arguments , argument
   | argument
   | \(\varepsilon\)}

argument
   \(\to\) type\_struct identifier
   
}\end{alltt}
\indent A seguir serão descritas quatro estruturas básicas da linguagem siC: comando com repetição, condicional, expressões matemáticas e expressões com pilhas e filas. A última contempla as operações de adicionar elemento no topo da pilha ou no fim da fila, "+", e remover do topo ou do início da fila, "-", onde o valor do elemento retirado é armazenado no último operando da expressão.
\begin{alltt}{\footnotesize

statements
   \(\to\) statements statement
   | \(\varepsilon\)

statement
   \(\to\) argument ';'
    | {\color{red}identifier ( identifiers ) ;}
    | {\color{red}identifier = identifier ( identifiers ) ;}
    | IF ( compare\_expression ) {\color{red}statement}
    | IF ( compare\_expression ) {\color{red}statement} ELSE {\color{red}statement}
    | WHILE ( compare\_expression ) {\color{red}statement}
    | identifier = {\color{red}assignment\_expression} ;
    | \textbf{identifier\_struct\_expression}
    | \{ statements \}
    
compare\_expression
   \(\to\) identifier\_struct compare\_assignment identifier\_struct
    
compare\_assignment
   \(\to\) == | != | <= | >= | {\color{red}<} | {\color{red}>}

assignment\_expressions
   \(\to\) assignment\_expression + term
    | assignment\_expression - term
    | term
    
term
   \(\to\) term * factor
    | term / factor
    | factor
    
factor
   \(\to\) identifier\_struct
    | {\color{red}' caractere '}
    | ( assignment\_expression )
    
\textbf{
identifier\_struct\_expression
   \(\to\) identifier = identifier + identifier ;
    | identifier = identifier - identifier ;
}		
}\end{alltt}

\section{Analisador Léxico}

\indent A principal tarefa do analisador léxico é examinar cada elemento do código fonte (variáveis, símbolos, números, etc), reconhecê-los com base em certos \textit{tokens} e classificá-los em grupos de lexemas. Além disso, ele pode realizar outras funções como eliminar espaços em brancos, tabulações, quebras de linhas e comentários; armazenar e acompanhar os números da linha e  coluna corrente no momento de sua execução; Informar mensagens de erro ou avisos de prevenção de erro diretamente ao usuário da linguagem \cite{book}.

\indent Existem três termos distintos bastante relacionados com analisador léxico. O primeiro, já citado anteriormente, é o \textit{token}, um par onde o primeiro elemento é o \textit{token name} (símbolo abstrato que representa um tipo de unidade léxica, como a palavra chave "while", por exemplo) e onde o segundo elemento é um atributo (informação adicional e opcional sobre o \textit{token}). O segundo termo é o \textit{pattern}, uma descrição, em forma de expressão regular, do um lexema de um token. No caso de um número inteiro, por exemplo, o \textit{pattern} seria uma sequência de um ou mais dígitos de 0 à 9. Por fim, o termo lexema significa uma sequência de caracteres no programa fonte que correspondem com o \textit{pattern} de um lexema específico, ou seja, cada lexema é uma instância de um token. No caso de existirem mais de uma correspondência, o lexema será a instância do \textit{token} cujo \textit{pattern} aparece primeiro no arquivo .lex.

\indent Esta seção apresenta o analisador léxico FLEX, utilizado neste trabalho, além de toda a descrição do arquivo .lex contruído a partir da gramática descrita no capítulo anterior.

\subsection{FLEX: The Fast Lexical Analyzer}

\indent Flex é uma ferramenta que gera um programa, chamado de \textit{scanner}, cuja função é identificar \textit{patterns} no código fonte. Ele recebe como entrada um arquivo de entrada especificados pelo usuário que serão reconhecidos a partir de expressões regulares mescladas com código em C (chamadas de descrição) no arquivo .lex \cite{flex}. Com o comando \textit{Flex nome\_do\_arquivo.lex}, um código fonte chamado \textit{lex.yy.c} é criado e nele existe uma função chamada \textit{yylex()}, a qual realiza de fato as operações do \textit{scanner}. Esse código, ao ser compilado corretamente com a flag \textit{-lfl} da biblioteca do flex, gera um arquivo objeto executável que recebe uma entrada qualquer e gera uma saída que depende do código em C que foi escrito no .lex (imprimir o lexema identificado, contabilizar o número de linhas, imprimir mensagem de erro léxico, etc).

\indent Neste projeto foram utilizadas duas variáveis globais muito úteis: \textit{yytext} e \textit{yyin}. A primeira contém o lexema que foi reconhecido como \textit{token}. Esta variável é modificada sempre que um novo lexema é identificado no código fonte. A segunda define como a entrada será lida, que pode ser tanto pela entrada padrão quanto por arquivo.

\subsection{Arquivo lex}

\indent O código fonte de extensão .lex é composto por três partes: definições, regras e código em C do usuário. Na seção de definições é onde são declarados nomes para certas expressões regulares, para facilitar a escrita das regras na próxima seção. Neste projeto foram feitas seis definições, apresentadas na Tabela \ref{TabelaDef}. A primeira representa um dígito apenas, de 0 à 9. A segunda é uma letra maiúscula ou minúscula e o símbolo \$, que em C pode compor o nome de um identificador. A terceira representa os símbolos de comparação entre dois números elementos (que serão do tipo char, inteiro e float). A quarta representa delimitadores do código siC, para finalizar um comanto, definir um escopo, etc. A quinta definição apresenta as quatro operações matemáticas básicas. A sexta e última apresenta uma expressão regular que especifica o formato de um identificador: ele deve começar com uma letra ou \$ e pode terminar com letras, \$s ou dígitos.

\begin{table}
 \centering
 \begin{tabular}{| c || r  c |}
  \hline
   & Nome & Definição \\
  \hline  \hline
  1 & digito & [0-9] \\
  \hline
  2 & letra & [a-zA-Z\$] \\
  \hline
  3 & comparison & == | != | <= | >= | < | > \\
  \hline
  4 & mark & . | ; | , | ' | \{ | \} | ( | ) \\
  \hline
  5 & operator & + | - | * | / \\
  \hline
  6 & id & [a-zA-Z\$][a-zA-Z\$0-9]* \\
  \hline
\end{tabular}
\caption{Tabela de definições}
\label{TabelaDef}
\end{table}
 
\indent A segunda parte do código lex é composto pelas regras que são um par de \textit{pattern} e ação que devem estar na mesma linha. A Tabela \ref{TabelaRegras} apresenta cada regra utilizada no projeto. No código fonte, todos os elementos da entrada que são identificados são imprimidos, exceto quebra de linha, espaços, tabulações e comentários, que são ignorados. Existem três variáveis contadoras que são utilizadas nas ações: \textit{lines}, que começa com um e é incrementada sempre que uma quebra de linha é reconhecida, e \textit{errors}, que começa com zero e é incrementada sempre que um erro léxico é encontrado. Ao longo da execução são imprimidos as descrições dos erros léxicos na tela e, ao final, o número total de erros.

\indent A ordem em que as regras estão é imporetante para o funcionamento correto do programa, pois se existir mais de uma correspondência de \textit{pattern} para um elemento, ele será identificado pela regra que aparecer primeiro. Neste projeto, as regras das keywords deve vir antes das regras dos identificadores, pois assim, se a seguinte entrada \textit{int x = 2;} for lida, por exemplo, o elemento \textit{int} será reconhecido como identificador e também tipo inteiro, porém a regra de keyword deverá identificá-lo.

\indent Os patterns contidos entre chaves foram definidos na seção de descrição. Em relação aos demais, segue abaixo a descrição de algumas expressões regulares. 

\begin{itemize}
  \item[[$\wedge\backslash$n]] : Reconhece tudo exceto espaço e quebra de linha (Exemplo: regra 3);
  \item[\{digito\}] : Reconhece apenas um dígito (Exemplo: regra 8);
  \item[\{digito\}*] : Reconhece zero ou mais dígitos (Exemplo: regra 6);
  \item[\{digito\}+] : Reconhece um ou mais dígitos (Exemplo: regra 5);
  \item[``abc''] : Reconhece a sequência de caracteres "abc" (Exemplo: regra 9);
  \item[``a'' $\vert$ ``b''] : Reconhece o caractere "a" ou o "b" (Exemplo: regra 8);
  \item[(?i:"AB")] : Reconhece as sequências "AB", "Ab", "aB", "ab" (Exemplo: regra 13);
  \item[.] : Reconhee qualquer elemento (Exemplo: regra 24).
\end{itemize}

\indent A terceira parte do código lex é composta por código em C, que define como será lida a entrada (por arquivo ou pela entrada padrão) e define também as ações tomadas por cada \textit{pattern}.

\begin{table}
 \centering
 \begin{tabular}{|c || c  l |} 
 \hline
   & \textit{Pattern} & Ação \\ [0.5ex] 
 \hline \hline
 1 & $\backslash$n	& Identifica uma quebra de linha e incrementa \\&& variável \textit{lines} para contagem de linhas \\ 
 \hline
 2 & [ $\backslash$t]+ 	& Identifica um ou mais espaço ou tabulação\\
 \hline
 3 & "//"[$\wedge\backslash$n]* & Ignora tudo a frente do comentário de \\&& uma linha "//" exceto quebra de linha \\
 \hline
 4 & \{digito\}+\{letra\}+\{digito\}* & {\color{red}Gera o erro identificado na linha \textit{lines}:} \\&& {\color{red}Sufixo inválido no número inteiro} \\
 \hline
 5 & \{digito\}+"."\{digito\}*\{letra\}+\{digito\}* & {\color{red}Gera o erro identificado na linha \textit{lines}:} \\&& {\color{red}Sufixo inválido no número float} \\
 \hline
 6 & \{digito\}+"."\{digito\}* & Identifica números float\\
 \hline
 7 & \{digito\}+ & Identifica números inteiros\\
 \hline
 8 & "'"(\{letra\}|\{digito\})"'" & Identifica valor para uma variável do \\&& tipo char, que pode ser uma letra ou dígito\\
 \hline
 9 & "''" & {\color{red}Gera o erro identificado na linha \textit{lines}:} \\&& {\color{red}Constante de caractere vazia} \\
 \hline 
 10 & \{comparison\} & Identifica símbolos de comparação \\&& entre dois elementos \\
 \hline
 11 & \{mark\} & Identifica pontuação e delimitadores de \\&& blocos e valores de char\\
 \hline
 12 & \{operator\} & Identifica operadores matemáticos \\
 \hline
 13 & (?i:"VOID")& Identifica palavra chave tipo void \\&& com as letras em caixa-alta ou caixa-alta \\
 \hline
 14 & (?i:"FLOAT")& Identifica palavra chave tipo float \\&& com as letras em caixa-alta ou caixa-alta \\
 \hline
 15 & (?i:"INT")& Identifica palavra chave tipo inteiro \\&& com as letras em caixa-alta ou caixa-alta \\
 \hline
 16 & (?i:"CHAR") & Identifica palavra chave tipo char \\&& com as letras em caixa-alta ou caixa-alta \\
 \hline
 17 & (?i:"QUEUE")& Identifica palavra chave tipo fila \\&& com as letras em caixa-alta ou caixa-alta \\
 \hline
 18 & (?i:"FIRST")& Identifica palavra chave que representa \\&&  primeiro elemento da fila com as letras em \\&& caixa-alta ou caixa-alta  \\
 \hline
 19 & (?i:"IF")& Identifica palavra chave condicional if \\&& com as letras em caixa-alta ou caixa-alta \\
 \hline
 20 & (?i:"ELSE")& Identifica palavra chave condicional else \\&& com as letras em caixa-alta ou caixa-alta \\
 \hline
 21 & (?i:"WHILE")& Identifica palavra chave do laço while \\&& com as letras em caixa-alta ou caixa-alta \\
 \hline
 22 & (?i:"RETURN")& Identifica palavra chave return, de retorno \\&& de função, com as letras em caixa-alta ou caixa-alta \\
 \hline
 23 & \{id\} & Reconhece um identificador cuja regra é: \\&& não é permitido conter símbolos além de \$, letras e \\&& dígitos e não é permitido começar a palara com dígito \\
 \hline
 24 & . & {\color{red}Gera o erro identificado na linha \textit{lines}:} \\&& {\color{red}\textit{Token} desconhecido} \\
 \hline
\end{tabular}
\caption{Tabela de regras}
\label{TabelaRegras}
\end{table}

\subsection{Erros Léxicos}

\indent Foram reconhecidos quatro erros léxicos em siC. Caso exista um elemento que comece com dígitos e termine com letras, o usuário será informado de que o sufixo de letras é inválido para um tipo inteiro (regra 4). Caso exista um elemento que comece com dígitos, tenha depois um ponto, e termine com letras, o usuário será informado de que o sufixo de letras é inválido para um tipo float (regra 5). Caso exista na entrada duas aspas simples, uma seguida da outra, o programa entende que entre eles deveria existir algum caractere que seria o valor que algum char, portanto o usuário será informado de que a constante de caractere está vazia (regra 9). Por fim, caso exista algum elemento não identificado na entrada, o usuário será informado (regra 24).

\subsection{Arquivos de Teste}

\indent Para testar o código foram criados dois arquivos de extensão .sic, um de acordo com as normas especificadas neste projeto (teste\_correto.sic), outro com todos os quatro tipos de erros léxicos (teste\_errado.sic).

%\section{Analisador Sintático}

%\subsection{Árvore sintática}

%\subsection{Tabela de símbolos}

\section{Analisador Semântico}

\indent A análise semântica utiliza da árvore sintática para checar a consistência da linguagem. Uma de suas obrigações mais importantes é a checagem de tipo. No caso do siC, existem várias restrições a serem consideradas:
\begin{itemize}
    \item Para adicionar um elemento A de tipo simples (char, int ou float) no fim da fila de um elemento struct B, a atribuição deve ser do tipo B = B + A, onde A deverá ter tipo compatível com o de B, ou seja, se B for fila de inteiros, A deve ser um inteiro;
    \item Para remover um elemento A de tipo simples (char, int ou float) do início da fila de um elemento struct B, a atribuição deve ser do tipo B = B - A, onde A deverá ter tipo compatível com o de B, ou seja, se B for fila de inteiros, A deve ser um inteiro;
    \item Nenhuma operação matemática (\textit{assignment\_expression}) pode conter um identificador B do tipo fila, apenas seu início, ou seja, B.FIRST.
\end{itemize}

\indent Um exemplo de código em siC é aprensentado a seguir. O programa adiciona três elementos numa fila de inteiros e depois eles são somados um a um e armazenados na variável \textit{sum}. Ao final, a variável \textit{lixo}, recém retirada da fila, é adicionada à \textit{sum}. Nesse sentido, o resultado final de sum deve ser 7. \\

\begin{lstlisting}[language=C]
VOID main () {
    QUEUE<INT> q;
    INT sum, INT lixo;

    q = q + 0;
    q = q + 1;
    q = q + 2;
    q = q + 3;    
    sum = 0;
    
    WHILE (q.FIRST != 0) {    
        sum = (sum + q.FIRST);
        q = q - lixo;
    }
    sum = sum + lixo;

    RETURN 0;
}

\end{lstlisting}

%\section{Geração de código intermediário}

% \section{Considerações finais}

\begin{thebibliography}{1}
\bibitem{book}
A.~V.~Abo, M.~S.~Lam, R.~Sethi, J.~D.~Ullman, \emph{Compilers - Principles, Techniques and Tools}
\hskip 1em plus
	0.5em minus 0.4em\relax 2nd ed. 1986
	
\bibitem{yacc}
ANSI C Yacc grammar, \url{http://www.quut.com/c/ANSI-C-grammar-y.html}, 18 12 2012.

\bibitem{flex}
Flex: The Fast Lexical Analyser, \url{http://flex.sourceforge.net/}, The Flex Project, 2008
\end{thebibliography}
%------------------------------------------------
\end{document}